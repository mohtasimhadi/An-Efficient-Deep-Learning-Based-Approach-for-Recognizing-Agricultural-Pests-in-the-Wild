In this study, the traditional method for pest classification was found to have several problems, such as difficulty in extracting features and a small size of the data sample. To address these issues, the transfer learning method and pre-trained CNNs were employed to classify the IP102 pest dataset. The following conclusions were drawn: The ConvNext model performed better than other pretrained models, achieving 76\% accuracy on the IP102 dataset. In the Ensemble approach, the combination of ViT and ConvNext achieved the highest accuracy at 78\%. The self-attention module in pretrained models like ViT is likely responsible for their superior performance. Cropping pest regions from the training set did not improve performance as it resulted in the loss of overall information, and retraining the trained model on both the original and cropped datasets did not help either. However, discarding images that did not contain pests improved overall performance by approximately 7-10\%. The Custom Model performed exceptionally well on the refined dataset where test set only contains the yolo identified images, achieving 85\% accuracy, which was the highest of all the experiments in this paper. It also performed slightly better on the original IP102 dataset.
 
