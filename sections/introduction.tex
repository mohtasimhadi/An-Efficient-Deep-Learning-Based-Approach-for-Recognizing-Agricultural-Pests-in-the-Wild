\subsection{Motivation}
The escalating global population and the ever-increasing demand for food have underscored the urgent need to enhance agricultural productivity while effectively managing the impact of agricultural pests on crop yields and quality. One significant factor that hampers agricultural productivity is the damage caused by insect pests to crops worldwide each year. Preventing such damages is crucial for commercial benefits, improved agricultural efficiency, and avoiding significant harvest losses. In addition to reducing the yield, crop pests can also inflict damage on machinery, equipment, soil, and infrastructure \cite{kandalkar2014classification}. Conventional methods employed to identify and classify agricultural pests in the wild are often characterized by time-consuming and expensive processes that heavily rely on expert knowledge. As a result, there is a compelling necessity for the development of an innovative and efficient approach to address this challenge.

By mitigating these problems, agriculture can drive economic growth while producing food with reduced resource consumption. Consequently, the recognition and classification of pests assume paramount importance in preventing crop damage. In recent years, there has been a notable surge of interest in automatic pests' classification \cite{xia2018insect}. Studies have shown that early detection and treatment can minimize damage to almost zero. However, this task is far from easy, as current methods for identifying insect pests are inefficient and expensive, relying heavily on the technical expertise of agricultural professionals.

Deep learning technology has emerged as a powerful tool in various fields, showcasing remarkable achievements, particularly in image-based tasks. This technology offers a promising opportunity to create a highly accurate and scalable solution for the recognition and classification of agricultural pests in diverse environmental conditions. By leveraging deep learning algorithms, it becomes possible to train models that can effectively analyze and interpret complex visual data, enabling accurate identification and classification of pests present in agricultural settings.

Exploiting deep learning technology offers immense potential in revolutionizing pest classification by providing efficient and cost-effective solutions. By leveraging the power of deep learning algorithms, researchers can develop robust models capable of accurately identifying and classifying insect pests, even in large-scale agricultural settings. The utilization of deep learning techniques can enhance the speed, accuracy, and scalability of pest detection and enable early intervention to minimize crop damage effectively.

Fortunately, advancements in deep learning techniques have paved the way for their adaptation across various domains, yielding remarkable results. Deep learning has achieved significant breakthroughs, particularly in image-based computer vision tasks such as image classification, segmentation, and detection \cite{reyes2015fine, he2016deep, huang2017densely}. The field of agriculture has also witnessed the successful application of deep learning methods in diverse areas, including plant identification, recognition, and classification \cite{reyes2015fine, dyrmann2016plant, zhang2018deep, ji20183d, lin2019fourier}, fruit counting \cite{chen2017counting}, and plant disease detection \cite{mohanty2016using}.

As the world grapples with the challenge of feeding an ever-growing population, ensuring food safety becomes imperative. Addressing the issue of insect pests, a major cause of crop damage, is crucial for improving agricultural productivity. Deep learning technology, with its significant achievements in image-based computer vision tasks, holds tremendous promise in the field of agriculture. By harnessing deep learning algorithms, researchers can enhance the recognition and classification of pests, leading to more efficient and cost-effective solutions for safeguarding global food production.


\subsection{Problem Statement}
Our research endeavors encompassed a comprehensive array of experiments conducted on the esteemed IP102 dataset \cite{wu2019ip102}, which is widely acknowledged in the field of agricultural pest recognition and classification. This dataset posed multifaceted challenges that required meticulous attention to detail in order to derive robust solutions. Notably, these challenges included the presence of pests within intricate backgrounds and foregrounds, the remarkable variability in their size, shape, color, and texture, and the marked imbalance in sample distribution among different pest species.

To surmount these hurdles and fortify the efficacy of our models, we devised a series of astute strategies that leveraged cutting-edge techniques. In particular, we delved into the realm of combined feature extraction, deftly merging ConvNext, an acclaimed architecture, with the innovative Vision Transformer (ViT) feature extractor. This synergistic amalgamation enabled us to harness the distinctive strengths of both architectures, culminating in an enhanced ability to capture a diverse range of features that are crucial for discerning and distinguishing between different pests with utmost precision.

Furthermore, we meticulously fine-tuned the linear dropout hyperparameter, painstakingly seeking an optimal equilibrium that would effectively mitigate the risks of overfitting while concurrently amplifying the generalization prowess of our models. This nuanced fine-tuning process imbued our models with a refined aptitude for accommodating diverse and unseen data samples with remarkable finesse.

Addressing the challenges emanating from complex backgrounds and foregrounds necessitated a meticulous exploration of transfer learning and fine-tuning methodologies. By adroitly harnessing the immense knowledge encapsulated within pre-trained models and adorning them with specialized adaptations to suit the intricacies of agricultural pest classification, we attained unprecedented strides in performance optimization. Furthermore, we judiciously employed ingenious data augmentation techniques such as Fmix and Cutmix to ingeniously alleviate the prevailing data/class imbalance and to augment the richness and diversity of our training samples. Additionally, we harnessed the potential of a cropped dataset, expertly extracting the regions of interest (ROI), and skillfully employing segmentation techniques to meticulously isolate and focus on the pivotal pest-related regions within the images.

The significance of attention-based methods was profoundly underscored in our research endeavors. We fervently explored multiple CNN-based models, adroitly integrating attention mechanisms into the core fabric of our models. A quintessential embodiment of this was manifested through the incorporation of the Convolution Block Attention Module (CBAM), an avant-garde innovation that facilitated an adaptive recalibration of feature maps, enabling our models to dynamically emphasize and prioritize the salient regions relevant to the pests under consideration. This judicious emphasis on pertinent information translated into a remarkable enhancement in the accuracy and acuity of our pest classification endeavors.

In our relentless quest for superior performance and minimized generalization errors, we judiciously employed ensemble methods, an ingenious approach that synergistically harnessed the collective wisdom of multiple models. Through astute employment of soft voting and hard voting mechanisms, we deftly amalgamated the predictions generated by the diverse array of models at our disposal. This ensemble-based confluence of perspectives endowed our models with heightened confidence and decisiveness, thereby effectively curtailing the risks associated with generalization errors and fostering superior classification outcomes.

\subsection{Organization}
Moreover, our research encompassed an incisive examination of the challenges posed by intra-class dissimilarity and inter-class similarity. By deftly stratifying pests based on their distinct life stages, namely early and late stages, we embarked upon a revelatory exploration of the metamorphic nature exhibited by most pests. This compelling insight enabled us to not only discern the prominent intra-class dissimilarities but also delineate the striking inter-class similarities that surfaced when considering pests throughout their life cycles. This meticulous classification based on pest stages served as a powerful tool for enhancing classification accuracy and, simultaneously, provided invaluable insights into the developmental. This insightful classification based on pest stages served as a powerful tool for enhancing classification accuracy and providing valuable insights into the developmental aspects of pests \cite{ji20183d, lin2019fourier}.

We have divided the paper into five more sections. The second section consists of literature review. We have read and reviewed other research works done in recent years in this domain and tried to find out the research problems. In the third section, namely methodology, we have discussed about our experiments and proposed methods. The fourth section consists of the results of our experiments and we tried comparing our results with other literature to find out where we stand. Finally we draw a conclusion to the paper and discussed what are the impacts of this work.